\documentclass{article}
\usepackage[utf8]{inputenc}
\usepackage{graphicx}
\usepackage{hyperref}
\usepackage{titlesec}

\title{KenyAInspire: An Adaptive Learning Platform. \\ Project Plan and Task Allocation}
\author{Author: Team HassDonn}
\date{Date: 28th September 2024}

\begin{document}

\maketitle

\section{Project Overview}
\subsection{Project Start Date}
27th September 2024

\subsection{Key Milestones}
\begin{itemize}
    \item Mini Hackathon: 11th October 2024
    \item Hackathon: 18th October 2024
\end{itemize}

\section{Task Breakdown}
\begin{enumerate}
    \item Donnelly: UI/UX Development (Accessibility Features focused UI)
    \item Alex: Backend Development (Databases, Integration and Management System)
    \item Hassan: AI Models Development (Machine Learning \& Natural Language Processing Models)
\end{enumerate}

\section{Project Timeline \& Weekly Breakdown}

\subsection{Week 1: 27th Sept – 30th Sept}
\subsubsection{Donnelly}
\begin{itemize}
    \item Create wireframes and UI mockups.
    \item Set up the basic ReactJS project structure.
\end{itemize}

\subsubsection{Alex}
\begin{itemize}
    \item Establish the backend architecture (Node.js).
    \item Design the database (MongoDB) schema for the platform.
    \item Implement authentication and user management system.
\end{itemize}

\subsubsection{Hassan}
\begin{itemize}
    \item Choose machine learning models and NLP algorithms.
    \item Research Adaptive Learning Algorithms and Multimodal NLP.
    \item Start the implementation of the personalization engine (ML).
\end{itemize}

\subsection{Week 2: 1st Oct – 10th Oct}
\subsubsection{Donnelly}
\begin{itemize}
    \item Develop the frontend interface for student, teacher, and admin portals.
    \item Add accessibility features (e.g resizable buttons/elements for the visually impaired etc)
\end{itemize}

\subsubsection{Alex}
\begin{itemize}
    \item Set up API endpoints to connect the backend to the frontend.
    \item Implement database queries for student data, learning paths, and performance tracking.
\end{itemize}

\subsubsection{Hassan}
\begin{itemize}
    \item Develop the initial ML and NLP models for adaptive learning.
    \item Implement basic text-to-speech and speech recognition features.
    \item Prepare a draft review of Adaptive Learning and Multimodal NLP for supervisor approval.
\end{itemize}

\subsection{Mini Hackathon: 11th Oct}
\begin{itemize}
    \item Showcase working UI, basic backend integration, and initial AI model functionality.
    \item Gather feedback and refine the platform.
\end{itemize}

\subsection{Week 3: 12th Oct – 17th Oct}
\subsubsection{Donnelly}
\begin{itemize}
    \item Polish the UI based on feedback from the mini hackathon.
    \item Complete all the accessibility featured UI elements.
    \item Add final interactive features.
\end{itemize}

\subsubsection{Alex}
\begin{itemize}
    \item Finalize all backend functionalities (data synchronization, offline mode).
    \item Ensure seamless integration with AI models.
\end{itemize}

\subsubsection{Hassan}
\begin{itemize}
    \item Ensure seamless integration with the Backend
    \item Complete the AI models for personalized learning and accessibility features.
    \item Finalize adaptive learning recommendations based on student progress.
    \item Refine NLP models for multimodal learning (text, voice).
\end{itemize}

\subsection{Hackathon: 18th Oct}
\begin{itemize}
    \item Final presentation with complete platform functionality (UI, Backend, AI integration).
\end{itemize}

\section{Algorithms \& Models}

\subsection{Adaptive Learning (Machine Learning Algorithms)}
We will use Reinforcement Learning for adapting learning paths based on student performance. The algorithm continuously assesses the student's progress and tailors the educational content accordingly. Key techniques:
\begin{itemize}
    \item Q-learning: Helps in selecting the most suitable learning material for each student by analyzing their prior responses.
    \item K-Nearest Neighbors (KNN): Used to group students with similar learning patterns to provide tailored content.
\end{itemize}

\subsection{NLP Algorithms (Multimodal NLP)}
For NLP, we will implement:
\begin{itemize}
    \item BERT (Bidirectional Encoder Representations from Transformers): Fine-tuned for contextually relevant translations and language comprehension.
    \item Text-to-Speech \& Speech Recognition: Models like Google's WaveNet will be used for text-to-speech, and DeepSpeech for speech recognition, providing accessibility for disabled students.
    \item Multimodal NLP: We'll explore CLIP (Contrastive Language-Image Pretraining) for integrating text and image content, making the platform more engaging through multimedia learning.
\end{itemize}

\section{Data Sources}
\begin{itemize}
    \item Kenyan Curriculum: Online resources, books, and open-source materials will be used for developing the content aligned with the Kenyan education system.
    \item Local Language Datasets: For NLP, datasets from African languages, especially Swahili, will be sourced from platforms like Masakhane and Nairobi-based linguistics research.
    \item Student Data: Historical student performance data (if available) or simulated data will be used for training the ML models.
\end{itemize}

\section{Potential Partnerships}
\begin{itemize}
    \item Edutab Africa: They focus on AI in education and could offer data and insights.
    \item St. Paul's University: To provide Student grades and Learning resources (from library)
\end{itemize}

\section{Project Document}
The project document will be refined continuously, with daily and weekly updates on the progress of algorithms, UI design, backend integration, and AI model improvements.

\end{document}